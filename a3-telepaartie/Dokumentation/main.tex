\documentclass[a4paper,10pt,ngerman]{scrartcl}
\usepackage{babel}
\usepackage[T1]{fontenc}
\usepackage[utf8x]{inputenc}
\usepackage[a4paper,margin=2.5cm]{geometry}
\usepackage{tikz}

% Die nächsten drei Felder bitte anpassen:
\newcommand{\Name}{Richard Wohlbold} % Teamname oder eigenen Namen angeben
\newcommand{\Einsendenummer}{???}
\newcommand{\Aufgabe}{Aufgabe 3: Telepaartie}

% Kopf- und Fußzeilen
\usepackage{scrlayer-scrpage}
\setkomafont{pageheadfoot}{\textrm}
\ifoot{\Name}
\cfoot{\thepage}
\chead{\Aufgabe}
\ofoot{Einsendenummer: \Einsendenummer}

% Für mathematische Befehle und Symbole
\usepackage{amsmath}
\usepackage{amssymb}

% Für Bilder
\usepackage{graphicx}

% Für Algorithmen
\usepackage{algpseudocode}

% Für Quelltext
\usepackage{listings}
\usepackage{color}
\definecolor{mygreen}{rgb}{0,0.6,0}
\definecolor{mygray}{rgb}{0.5,0.5,0.5}
\definecolor{mymauve}{rgb}{0.58,0,0.82}
\lstset{
  keywordstyle=\color{blue},commentstyle=\color{mygreen},
  stringstyle=\color{mymauve},rulecolor=\color{black},
  basicstyle=\footnotesize\ttfamily,numberstyle=\tiny\color{mygray},
  captionpos=b, % sets the caption-position to bottom
  keepspaces=true, % keeps spaces in text
  numbers=left, numbersep=5pt, showspaces=false,showstringspaces=true,
  showtabs=false, stepnumber=2, tabsize=2, title=\lstname
}
\lstdefinelanguage{JavaScript}{ % JavaScript ist als einzige Sprache noch nicht vordefiniert
  keywords={break, case, catch, continue, debugger, default, delete, do, else, finally, for, function, if, in, instanceof, new, return, switch, this, throw, try, typeof, var, void, while, with},
  morecomment=[l]{//},
  morecomment=[s]{/*}{*/},
  morestring=[b]',
  morestring=[b]",
  sensitive=true
}

% Diese beiden Pakete müssen als letztes geladen werden
%\usepackage{hyperref} % Anklickbare Links im Dokument
\usepackage{cleveref}

% Daten für die Titelseite
\title{\Aufgabe}
\author{\Name\\Einsendenummer: \Einsendenummer}
\date{\today}



\begin{document}

\maketitle
\tableofcontents

\section{Lösungsidee}
Die Zustände und Transformationen der Biberverteilungen können als gerichteter Graph dargestellt werden.
Dabei repräsentiert ein Knoten im Graph eine Biberverteilung.
Eine Kante von Knoten $K_1$ zu Knoten $K_2$ existiert, falls die Biberverteilung des Knotens $K_1$ mit dem \textit{Telepaarteur} in die Biberverteilung des Knotens $K_2$ umgewandelt werden kann.
Die Anordnung der Füllmengen der Behälter ist egal, da sich die Anordnung $(x,y,z)$ genauso lösen lässt wie die Anordnung $(x,z,y)$.
Somit repräsentiere ich die Füllmengen an einem Knoten als Menge von drei Zahlen.
Enthält die Menge an einem Knoten den Wert 0, beträgt die Leerlauflänge 0.
Ansonsten beträgt die Leerlauflänge die geringste Anzahl von Kanten zu einem Knoten, dessen Menge eine 0 enthält.

Ein Beispielgraph ist in Abbildung \ref{fig:graph-idee} zu sehen.

\begin{figure}
\centering
\begin{tikzpicture}
    [auto=left,every node/.style={circle,fill=blue!20}]
    \node (n1) at (0,4) {\{2,4,9\}};
    \node (n2) at (-2,2) {\{4,4,7\}};
    \node (n3) at (0,2) {\{1,2,8\}};
    \node (n4) at (-2,0) {\{0,7,8\}};
    \node (n5) at (0,0) {...};
    \node (n6) at (2,2) {...};
    \draw [->] (n1) -- (n2);
    \draw [->] (n1) -- (n3);
    \draw [->] (n2) -- (n4);
    \draw [->] (n3) -- (n5);
    \draw [->] (n1) -- (n6);
\end{tikzpicture}
\label{fig:graph-idee}
\caption{Beispiel aus der Aufgabe, als Graph repräsentiert}
\end{figure}

Ein Aufgabenmerkmal ist, dass der eingeführte Graph gerichtete Zyklen enthalten kann, d.h. man kann über Telepaartie zum gleichen oder einem vorherigen Zustand gelangen.
Ein Beispiel dafür ist die Biberverteilung ${2,4,4}$, die erneut durch Umfüllen zweier Biber in den Behälter, in dem sich zwei Biber befinden, erreicht werden kann.
Diese Eigenschaft muss bei der Lösung der Aufgabe berücksichtigt werden.

Um das Problem zu lösen und den kürzesten Weg zu einem Knoten zu finden, der eine $0$ enthält, wende ich eine \textit{Breadth-First-Search} an.
Diese durchquert den Graphen gleichmäßig, indem stets der ununtersuchte Knoten untersucht wird, der den kürzesten Weg zum Wurzelknoten, also der Ausgangsverteilung, besitzt.
Wird ein Knoten gefunden, der eine $0$ enthält, ist der Prozess beendet, denn man kann sich sicher sein, dass es keinen kürzeren Weg zu einem Knoten gibt, der eine $0$ enthält.

Breadth-First-Search wird mit einer \textit{Queue}, also First-In-First-Out-Datenstruktur, umgesetzt.
Wird ein Knoten untersucht, so wird überprüft, ob die Verteilung am zu untersuchenden Knoten eine 0 enthält. Ist dies der Fall, ist die Prozedur beendet.
Ansonsten werden alle Knoten, zu denen Kanten vom untersuchten Knoten aus bestehen und die noch nicht untersucht wurden, ans Ende der Schlange gesetzt.
Der erste zu untersuchende Knoten ist der Wurzelknoten (also die zu untersuchende Biberverteilung).
Eine leere Menge wird anfangs initialisiert, der nach und nach alle untersuchten Knoten hinzugefügt werden.

\section{Umsetzung}
Hier wird kurz erläutert, wie die Lösungsidee im Programm tatsächlich umgesetzt wurde. Hier können auch Implementierungsdetails erwähnt werden.

\section{Beispiele}
Genügend Beispiele einbinden! Eigene Beispiele sind sehr gut! Und die Beispiele sollte diskutiert werden.

\section{Quellcode (ausschnittsweise)}
Unwichtige Teile des Programms müssen hier nicht abgedruckt werden.

\end{document}
